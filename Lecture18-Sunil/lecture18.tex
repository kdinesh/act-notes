\documentclass[11pt]{article}
\usepackage{amsmath,amssymb,amsfonts,amsthm,complexity}
\usepackage{verbatim}
\newcommand{\CYCLE}{{\sf CYCLE~}}
\newcommand{\FPSPACE}{{\sf FPSPACE~}}
\usepackage{tikz}
\usetikzlibrary{calc,through,backgrounds,decorations.pathmorphing}

\input{../stylefile/preamble.tex}
\begin{document}
\lecture{18}{Proof of Toda's Theorem}{Feb 07, 2012}{ Jayalal Sarma M.N.}{Sunil K
  S.}  \theme{Between P and PSPACE}
\lectureplan{In this lecture we will be concluding the lectures on the
  theme of contrasting the power of counting to that of
  alternations. Today we will be proving the interesting result that
  $\PH \subseteq \P^{\#\P}$. To do this, first using the
  Valiant-Vazirani lemma we will show that $\PH \subseteq
  \BP(\parity\P)$.}
\section{Valiant-Vazirani Lemma}
We have already saw Valiant-Vazirani theorem which stated that :
\begin{lemma} Valiant-Vazirani Theorem: There exists a probabilistic
polynomial-time algorthm f such that for every n-variable boolean
formula $\varphi$
\begin{eqnarray*}
  \varphi\in SAT\Rightarrow Pr[f(\chi)\in USAT] &\geq& \frac{1}{8n} \\
  \varphi\notin SAT\Rightarrow Pr[f(\chi)\in SAT] &=& 0
\end{eqnarray*}
\end{lemma}
But to prove that $\PH \subseteq \BP(\parity\P)$ we will need
amplified version of this lemma.
\begin{lemma}
Valiant-Vazirani lemma (Amplified version): There exists a randomized
algorithm which produce $\varphi$ from a given boolean formula $\phi$
such that for a polynomila $q(n)$:
\begin{eqnarray*}
  x\in L \Rightarrow Pr[\varphi\in \parity SAT] &\geq&
  \left(1-\frac{1}{2^{q(n)}}\right) \\ 
  x\notin L \Rightarrow Pr[\varphi\notin \parity SAT] &=& 1
\end{eqnarray*}
\end{lemma} In the above construction, $\varphi=\phi\land h$, where
$h$ is a hash function. Here $\varphi$ is independent on $\phi$ or its
structure.  To find $h$, chosen from a good hash family, we just need
to know the domain and range, which is $2^n$. Here $h$ depends only on
$n$. Let $\tau(x,y)$ denote the choice of $h$ from the hash family
based on the random string $y$.  Now we modify the lemma statements
given above as:
\begin{eqnarray*}
  (\exists \phi) \phi\in  SAT\Rightarrow
  Pr_y[\displaystyle\parity_x\phi\land\tau(x,y)] &\geq&
  \left(1-\frac{1}{2^{q(n)}}\right) \\
  (\lnot\exists \phi) \phi\notin  SAT\Rightarrow
  Pr_y[\displaystyle\parity_x\phi\land\tau(x,y)] &=& 1
\end{eqnarray*}
where $\displaystyle\phi_x$ represents the parity of number of
satisfying assignments.

Observations: Existing of a satisfying assignment for $\phi$ is
equivalent to saying $\phi\in SAT$. Here we can talk about any
$\phi$. That is $\phi$ can even have $\exists$ or $\forall$
quantifiers. This gives a better handle since construction is
completely oblivious of what $\phi$ is.

Claim: $\PH\subseteq\BP(\parity\P)$
\begin{proof} Proof by induction on the number of alterations, $k$.\\
  Basis: for $k=0$, we already have the result
  $\NP\subseteq\BP(\parity\P)$.\\ Assume the result for $k$. ie;
  $\Sigma_k$ or $SAT_k$ $\in \BP(\parity\P)$.\\ We need to prove that
  $SAT_{k+1}\in \BP(\parity\P)$.
$$\phi\in SAT_k\Rightarrow Pr[\parity_z\phi\land\tau(x,z)]\geq(1-\frac{1}{2^{q(n)}})$$
$$\phi\notin SAT_k\Rightarrow Pr[\parity_z\phi\land\tau(x,z)]=0$$
Now, any $\phi '\in SAT_{k+1}$ can be written as $\phi
'=\exists\sigma$ where $\sigma\in SAT_k$
$$\phi '\in SAT_{k+1}\Rightarrow\exists\sigma\in SAT_k\mbox{ with } \sigma\in\Pi_k$$

Let $\sigma '=\exists(\lnot\varphi)$ where $\sigma=\lnot\varphi$\\
$$\lnot(\forall\varphi)\Rightarrow\exists\sigma\Rightarrow Pr[\sigma '\in\parity SAT]\geq (1-\frac{1}{2^{q(n)}})$$
$$\forall\varphi\Rightarrow\lnot\exists\sigma\Rightarrow Pr[\sigma '\notin\parity SAT]\geq (1-\frac{1}{2^{q(n)}})$$

$$\forall\varphi\Rightarrow Pr[\sigma ''\in\parity SAT]\geq (1-\frac{1}{2^{q(n)}})$$
$$\lnot\forall\varphi\Rightarrow Pr[\sigma ''\notin\parity SAT]\geq 1$$
$$\varphi\rightarrow\varphi\land(h(y)=0^k)\rightarrow\varphi\land(h(y)=0^{k_1}\land h(x)=0^{k_2})$$
$$\phi=\exists\forall\varphi$$

\end{proof}
\section{Toda's Theorem: $\PH\subseteq\P^{\#\P}$} Any problem in $\PH$
can be solved by a $\P$ machine by making queries to a $\#\P$ machine.
\begin{lemma}
\label{lemma1} If $\mathcal{A}\in \parity\P$ then $\exists B$ such
that for any polynomial $q$ and input $x$ of length $n$,
$$x\in \mathcal{A}\Rightarrow (\#(x,y)\in B)\equiv -1 \mod 2^{q(n)}$$
$$x\notin \mathcal{A}\Rightarrow (\#(x,y)\in B)\equiv 0 \mod 2^{q(n)}$$

Now we can state the lemma as, Let $\mathcal{A}\in \parity\P$. Then
for any polynomial $q$, there exists a polynomial-time NTM M such that
for any input $x$ of length $n$,
$$x\in \mathcal{A}\Rightarrow (\chi_M(x))\equiv -1\mod 2^{q(n)}$$
$$x\notin \mathcal{A}\Rightarrow (\chi_M(x))\equiv 0 \mod 2^{q(n)}$$
Here $\chi_M(x)$ denotes the number of accepting computations of $M$
on $x$.
\end{lemma}
\begin{proof} Let $M_1$ be a polynomial time NTM such that
$$\chi_{M_1}\equiv 1\mod 2\mbox{ if } x\in \mathcal{A}$$ %and
$$\chi_{M_1}\equiv 0\mod 2 \mbox{ if }x\notin \mathcal{A}$$ 

In other words,
$$x\in \mathcal{A}\Rightarrow (\#acc_M(x)) \mbox{ is odd}$$
$$x\notin \mathcal{A}\Rightarrow (\#acc_M(x))\mbox{ is even}$$


Define another polynomial time NTM $M_2$ that repeats $M_1$ on $x$ a
number of times such that
$$x\in \mathcal{A}\Rightarrow (\#acc_{M_2}(x)) \mbox{ is odd}$$
$$x\notin \mathcal{A}\Rightarrow (\#acc_{M_2}(x))\mbox{ is even}$$

Note: Given two NDTMs $M_1$ and $M_2$, we know how to get an $M_3$
with:
\begin{itemize}
\item $\#acc_{M_3}(x)=\#acc_{M_1}(x)+\#acc_{M_2}(x) $: By running
$M_1$ and $M_2$ in parallel.
\item $\#acc_{M_3}(x)=\#acc_{M_1}(x)\times\#acc_{M_2}(x) $: By rnning
$M_1$ and $M_2$ one after other.
\end{itemize}
\begin{comment} The following psuedocode represents the working of
$M_2$ in a recursive form.  Non-deterministic Turing machine:
$M_2(<x,i>)$\\ //$i$: Number of times $M_2$ executes $M_1$.\\
\begin{enumerate}
\item If $i=0$, then simulate $M_1(x)$
\item Else if $i>0$
	\begin{enumerate}
	\item Nondeterministically choose an integer
$m\in\{1,2,\cdots,7\}$
	\item If $m\leq 4$ then let $k=3$ else let $k=4$
	\item For $j=1$ to $k$ do
		\begin{enumerate}
		\item recursively run $M_2(<x,i-1>)$
		\item if $M_2(<x,i-1>)$ rejects then reject and halt
		\end{enumerate}
	\item Accept and halt
	\end{enumerate}
\end{enumerate}
\end{comment}

Let $f(x,i)=\chi_{M_2}(<x,i>)$. It if clear that $f$ satisfies the
recurrence relation given below:
\begin{eqnarray}
\label{rec} f(x,i+1)=3f(x,i)^4+4f(x,i)^3, i\geq0
\end{eqnarray} From equation ~\ref{rec}
$$f(x,0)\mbox{ is even }\Rightarrow f(x,i)\equiv 0\mod 2^{2^i}$$
$$f(x,0)\mbox{ is odd }\Rightarrow f(x,i)\equiv -1\mod 2^{2^i}$$
Now from the above analysis,
$$x\in \mathcal{A}\Rightarrow (\chi_M(x)\equiv  -1\mod 2^{2^{\log q(n)}}\equiv -1\mod 2^{q(n)}$$
$$x\notin \mathcal{A}\Rightarrow (\chi_M(x)\equiv  0\mod 2^{2^{\log q(n)}}\equiv 0\mod 2^{q(n)}$$
\end{proof}


\begin{theorem} $\BP(\parity\P)\subseteq \P^{\#\P}$
\end{theorem}
\begin{proof} $L\in \BP(\parity\P)$ means there exists a set
$\mathcal{A} \in \parity\P$ and a polynomial $p$ such that for all
$x$,
$$x\in L \Rightarrow Pr_y[(x,y)\in \mathcal{A}]\geq \frac{2}{3}$$
$$x\notin L \Rightarrow Pr_y[(x,y)\in \mathcal{A}]\leq \frac{1}{3}$$
where $y$ ranges over all strings of length $p(|x|)$.

By lemma ~\ref{lemma1}, if $\mathcal{A} \in \parity\P$, there exists a
polynomial-time NTM M such that for all $(x,y)$, with $|x|=n$ and
$|y|=p(n)$
$$\chi_M(<x,y>)\equiv -1\mod 2^{p(n)}, \mbox{ if } <x,y>\in\mathcal{A}$$
$$\chi_M(<x,y>)\equiv 0\mod 2^{p(n)}, \mbox{ if } <x,y>\notin\mathcal{A}$$


Let $g(x)$ and $h(x)$ are two functions defined as,
\begin{eqnarray*} g(x)=|\{y:|y|=p(|x|),(x,y)\in \mathcal{A}\}|\\
h(x)=\displaystyle\sum_{|y|=p(|x|)}\chi_M(<x,y>)
\end{eqnarray*} Then for any $x$ of length $n$,
\begin{eqnarray*}
h(x)&=&\displaystyle\sum_{<x,y>\in\mathcal{A}}\chi_M(<x,y>)+\displaystyle\sum_{(<x,y>\notin\mathcal{A})}\chi_M(<x,y>)\\
&=&(\displaystyle\sum_{<x,y>\in\mathcal{A}}(-1)+\displaystyle\sum_{(<x,y>\notin\mathcal{A})}0)\mod
2^{q(n)}\\ %&\equiv&(g(x).(-1)+(2^{p(n)}-g(x)).0)\mod 2^{p(n)}\\
&\equiv&(g(x).(-1)+(2^{p(n)}-g(x)).0)\mod 2^{p(n)}\\
&\equiv&(-g(x))\mod 2^{p(n)}
\end{eqnarray*} Construct a machine $N$ that has $h(x)$ accepting path
(Just guess a $y$ and run $N$). Now make a query to a $\#\P$ machine
to compute $h(x)$. By having $q(n)$ sufficiently large, ie.,
$2^{q(n)}>2{p(n)}$ we can compute $g(x)=(2^{p(n)}-h(x))$.

We know $x\in L$ if and only if $g(x)>2^{p(n)-1}$. Since $g(x)$ can be
computed from $h(x)$ and $p(n)$ it is clear that we can decide whether
$x\in L$ from $h(x)$. The function $h$ is in $\#\P$, we can define an
NTM $M_1$ that on input $x$ first nondeterministically guesses a
string $y$ of length $p(|x|)$ and then simulate $M$ on $<x,y>$.
Hence, $L\in P^{\#\P}$.
\end{proof}

\begin{comment} We saw $(\#P=FP)\Longrightarrow (P=PP)$ in decision
world. Now what will be the notion of reduction in counting world?
\begin{description}
\item{\textbf{Attempt 1:}} Let A,B be two languages. In decision world
$A\leq ^p_m B$ if and only if
	\begin{enumerate}
		\item $x\in A\Longleftrightarrow \sigma\in B$.
		\item $\sigma : \Sigma ^*\rightarrow\Sigma^*$ is
polynomial time computable.
	\end{enumerate} Translating the above statements into counting
scenario.
	\begin{enumerate}
		\item $\sigma$ is polynomial time computable.
		\item $f(x)=g(\sigma(x))$.
	\end{enumerate} We know, $\#SAT$ and $\#CYCLE$ must also be
hard. We must be able to show with the above discussed counting
scenario that
	\begin{center} $\#SAT(\phi)=\#CYCLE(\sigma(\phi))$
	\end{center} Now if such a reduction exists, then we can solve
$SAT$ by checking if $\#SAT(\phi)=\#CYCLE(\sigma(\phi))$, which can be
done in polynomial time.  Here the problem is that one language is in
$P$. But if we impose this strictness then in $\#P$ only $NP-Complete$
problems will be hard.  But we already know $\#CYCLE$ is also hard.
\item{\textbf{Attempt 2:}} $f\leq g$, $f\in FP^g$, if there exists a
functional oracle turing machine (instead of going between
$q_{yes}/q_{no}$, it has an output tape to write a value as
output.)with queries to $g:\Sigma^*\rightarrow$ $\mathcal{N}$ that
given any $x\in \Sigma^*$ can compute $f(x)$.  \\A function $g$ is
$P-hard$ if $\forall f\in \#P, f\leq_m^p g$. $g$ is complete for $\#P$
if $g\in \#P$ and $g$ is $\#P-hard$.
\end{description}
\begin{theorem} $\#SAT$ is $\#P-Complete$
\end{theorem}
	\begin{enumerate}
		\item $\#SAT\in \#P$: $\exists$ a non-deterministic
turing machine polynomial time which guesses assignments and verifies
the same.
		\item $\forall f\in \#P, f\leq_m^p \#SAT$: Let $f\in
\#P$ via a machine $M$ such that $\forall x \in \Sigma^*,
f(x)=\#acc_M(x)$
	\end{enumerate}

\begin{description}
\item \textbf{Cook-Levin Theorem - Revisited.}
	\begin{center} $L\in NP \Longrightarrow L\leq^p_m SAT$.
	\end{center} Let $L\in NP$, via machine $M$ such that $x\in L
\Longrightarrow M$ has an accepting path.  We construct $\phi_{(M,x)}$
from computation history such that $ M$ has an accepting path implies
$ \phi_{(M,x)}$ is satisfiable.  So for every accepting path, we have
a satisfying assignment.  \\ ie., There is a one-to one mapping
between the set of accepting paths and the set of satisfying
assignments. The certificates correspods to the first set is the
choice of non-deterministic bits and for the second set is the
satisfying assignments. Such reductions are called \emph{Parsimonious
reductions}.  \\ \emph{Note:} To prove NP-completeness, reduction need
not be parsimonious.
\end{description}
\section{Perfect Matching in a graph}

	\begin{definition}: Perfect Matching:

	A bipartite graph $(X,Y,E)$ has a perfect matching if
$\exists$ a subset of edges ($S$) such that each vertex has exactly
one edge from $S$ incident on it.
	\end{definition} \vspace{5mm}
	\begin{definition}: Permanent of a Matrix For an $n\times n$
matrix A, \\ Determinant of A, $Det(A)=\displaystyle\sum_{\sigma\text{
is a permutation}\in S_n} (-1)^{sign(\sigma)}\prod_{i=1}^n
A_{i,\sigma(i)}$ \\ Permanent of A,
$Per(A)=\displaystyle\sum_{\sigma\in S_n} \prod_{i=1}^n
A_{i,\sigma(i)}$
	\end{definition}
	\begin{description}
	\item \textbf{Claim}: For a bipartite graph $G$, with $A$
being the adjacency matrix in which rows are indexed by $x$ and
columns are indexed by $y$. (Note that here $|x|=|y|$, otherwise graph
doesnot have a perfect match.)
	\end{description} For any $\sigma$,
$\displaystyle\prod_{i=1}^n A_{i,\sigma(i)}=1$ if and only if $\sigma$
gives a perfect matching in $G$.  \\ \emph{Note:}
$A_{i,\sigma(i)}=1\Longleftrightarrow (i,\sigma(i))\in E$. Also the
product is 1, if it holds for all.  \\ Now we can figure out whether
there exixts a perfect matching or not. Checking if $G$ has a perfect
matching is in $P$. Testing if $Per(A)$ is non-zero is also in $P$.

\end{comment}
\end{document}
