
\newcommand{\sP}{\#\P} %% The class #P as it is not defined in complexity.sty
\Lecture{Akshay Degwekar, Devanathan.T}{Jan 17, 2012}{8}{Permanent Computation is \sP-Complete}

\newcommand{\per}{{\sf Per}} % Permanent is defined as an operator.
\newcommand{\wt}{{\sf Weight}} % Weight is defined as an operator.
\newcommand{\integer}{\mathbb{Z}}

In this lecture, we define the permanent of a matrix and study the complexity of computing the permanent for various classes of matrices. Finally we prove a theorem by Valiant that $\per(A)$ is \sP-Complete.

\section{Definitions}
\begin{definition}
\textbf{Permanent } Given a matrix $A_{n \times n}$, Let $S_n$ be the set of
all permutations of $\{1,2,\ldots,n\} = [n]$. Then
\begin{equation}
\per(A) = \sum_{\sigma \in S_n } \prod_{i=1}^n A_{i,\sigma(i)} 
\end{equation}
\end{definition}

Earlier, we have seen a characterization of the permanent as the number of perfect matchings in a graph G. To prove the result, we characterize the permanent using \textbf{Cycle Covers}.

\begin{definition}
\textbf{Cycle Cover } A Cycle Cover $C$ of a directed graph G is a set of pairwise disjoint simple cycles such that each vertex lies in exactly one cycle in $C$.

Note that we allow self-loops as simple cycles.

\xymatrix{
		  & 3 \ar[dr]&	            &   	&	7\ar[dr] \\
2 \ar[ur]& 		 &4\ar[dl]\ar[r]&6\ar[ur]  && 8\ar[dl] \\
&1 \ar[ul]&&&5\ar[ul]&\\}
Has a cycle cover $\big{\{}\{1,2,3,4\},\{5,6,7,8\}\big{\}}$.

\begin{definition}
\textbf{Weight of a Cycle Cover} Given a Graph $G(V,E)$, let $C={C_1, C_2, \dots C_l}$ be any cycle cover
\begin{equation}
\textbf{Weight of Cycle - } \wt(C_i) = \prod_{e\in E}w(e)
\end{equation}
\begin{equation}
\textbf{Weight of Cycle Cover - } \wt(C_i) = \prod_{c\in C}\wt(c)
\end{equation}
\end{definition}


%%That is, In a graph $G(V,E)$, $C = {C_1, C_2, \dots C_l}$ is a cycle cover iff $\forall i,j \quad C_i\cap C_j = \phi$ and $v\in V, \, \exists ! j $ such that $•$ $C_j$ is a 

\end{definition}

\begin{lemma}
Let $G(V,E)$ be a graph with adjacency matrix $A$. 
$A \in \mathbb{Z}^{n\times n}$. Then 
\begin{equation}
	\per(A) = \sum_{\substack{C \text{ is a} \\ \text{Cycle Cover}}} \wt(C)
\end{equation}
\end{lemma}
\begin{proof}
Consider a single term from the permanent - $\sum^n_{i=1}A_{i,\sigma(i)}$.

We can view $\sigma$ as a cycle cover as follows. 

First we decompose $\sigma$ into cycles of the form $a, \sigma(a),
\sigma^2(a)\dots \sigma^k(a) = a$. Now each cycle in the permutation can be
viewed as a cycle in the graph G, the edges being 
\[ \big{\{} (1, \sigma(1)),(2, \sigma(2)), \dots (n, \sigma(n)) \big{\}} \]
This is the required cycle cover. 

For the other direction, some of the cycle covers generated from permutation
might not be valid as some edges are absent. In that case, we just see that 
their weight is 0 as $A_{i,j} = 0$ for the missing edge $(i,j)$. And in the 
case that it is a valid cover, $\wt(C) = \prod_{i\in[n]}A_{i,\sigma(i)}$, because both of them are exactly the product of the corresponding edges.

This shows a bijective correspondence between the permutations and the cycle
covers which completes the proof.
\end{proof}

%%%%%%%%%%%%%%%%Some explanation and notes are required here.

\begin{exercise}
Let $\mathbb{Z}_+$ denote the non negative integers.
Using the previous construction involving cycle covers, 
show that for $A\in \mathbb{Z}_{+}^{n\times n}, \per(A)\in \FP^{\sP}$.

% This problem is also a part of the first problem set.
\end{exercise}


\begin{exercise}
Is the reduction from $\SAT$ to $3\SAT$ parsimonious? If yes, show that
$\#3\SAT$ is $\sP$-complete. 
\end{exercise}

\section{Proof of harndess of Permanent computation}
%Now, we move towards the final part of the lecture - the result by Valiant, that $Per$ is $\sP-Hard$. The proof given here is due to Del.... %add the citations here for both the papers.
In this section, we show that computing permanant of a $0-1$ matrix is $\sP$
complete. The result is due to Valiant~\cite{valiant79}. The proof presented
here is from Dell. et.al~\cite{dell12}. 

We will show this by using a gadget construction.

\begin{theorem}
$\#3\SAT\in \FP^{\per}$
\end{theorem}
\begin{proof}
Consider a formula $\phi $ in $3\SAT$. $\phi = C_1\wedge C_2 \dots C_m$ where $C_i = l_{i,1}\vee l_{i,2} \vee l_{i,3}$.

We will construct a directed graph $G$ such that $A_G$ is the adjacency matrix of $G$, such that $A_G \in \{-1,0,1\}^{n\times n}$ and also, $\per(A_G)=(-2)^k(\#\phi)$ where $(\#\phi)$ is the number of satisfying assignments of $\phi$ and $k$ is a quantity that we specify later.

\begin{lemma}
Let $\#x$ be the number of times a variable $x$ occurs in $\phi$. 

Then $\phi$ can be converted to $\phi'$ such that $\forall x$, $k = \#x=\# \overline{x}$ and $\#phi = \# phi'$
\end{lemma}

\begin{proof}
 We first observe that if $\# x \not = \# \overline{x}$. Then this imbalance can be removed by adding terms of the form $(x\vee x\vee \overline{x})$ and/or $(x\vee \overline{x}\vee \overline{x})$ because they add or decrease the relative number of $x$ compared to $\overline{x}$.
 
 Now, we assume that $\forall x, \#x = \# \overline{x}$.
 Now to compensate for relative differences between $x, y$ we add terms of the form $(x\vee \overline{x}\vee \overline{y})\wedge (x\vee \overline{x}\vee \overline{y})$ to increase the number of $x$ and the other way to decrease.
 
 And we just note that the number of solutions is invariant because each of the added terms are always true.
 
 This completes the lemma.
 \end{proof}
  
  We will be constructing the graph from three gadgets - Variable Gadget, Clause Gadget and the Equality Gadget.

%%% The gadgets  
\begin{figure}[h!]
\centering
%%%%%%%Variable Gadget
\begin{subfigure}[b]{0.3\textwidth}
\xymatrix{ 
	{\bullet}  \ar @/_/ [dd]_x  \ar @/^/ [dd]^{\overline{x}} \\ \\
	{\bullet} \ar [uu]}
	\caption{Variable Gadget}
\end{subfigure} 
%%%%%%%Clause Gadget
\begin{subfigure}[b]{0.3\textwidth}
\centering
 \xymatrix{ 
	 & {\bullet}  \ar @/^/ [d]   \ar @/_/ [ddl]_{\overline{l_1}} & \\ 
	 & {\bullet}  \ar @/^/ [u]  \ar @/^/ [dl]  \ar @/^/ [dr] & \\
{\bullet}\ar @/_/ [rr]_{\overline{l_3}} \ar @/^/ [ur] & & {\bullet}  \ar @/^/ [lu] \ar @/_/ [uul]_{\overline{l_2}} \\	
	}
	\caption{Clause Gadget}
\end{subfigure}
%%%%%%%Equality Gadget
\begin{subfigure}[b]{0.3\textwidth}
\centering
 \xymatrix{ 
	\ar @/_/[ddr] & & {\bullet} \ar@(ul,ur)^{-1} \ar @/_/[ddr] \ar @/_/ [ddl]& & \ar @/^/[ddl] \\ 
	\\
	\ar @/^/[r] & {\bullet}  \ar@(dl,dr)_{1}\ar @/_/ [uur]  \ar @/_/ [rr] & &  {\bullet}  \ar @/_/[uul]  \ar @/_/[ll] \ar@(dl,dr)_{1} & \ar @/_/[l] }
	
	\caption{Equality Gadget}
\end{subfigure}
\caption{The Gadgets}
\label{fig1Gadgets}
\end{figure}

Now, we consider the construction of the graph.

For each variable pair $(x, \overline{x})$ we have one variable gadget. Each clause has a Clause Gadget and the equality gadget is used to join each variable with all the clauses the variable is in.

The equality gadget is represented as a black box as follows -
\begin{figure}[h!]
\centering
\begin{subfigure}[b]{0.3\textwidth}
\centering
 \xymatrix{ 
	\ar @/_/[ddr] & & {\bullet} \ar@(ul,ur)^{-1} \ar @/_/[ddr] \ar @/_/ [ddl]& & \ar @/^/[ddl] \\ 
	\\
	\ar @/^/[r] & {\bullet}  \ar@(dl,dr)_{1}\ar @/_/ [uur]  \ar @/_/ [rr] & &  {\bullet}  \ar @/_/[uul]  \ar @/_/[ll] \ar@(dl,dr)_{1} & \ar @/_/[l] }
	
	\caption{Equality Gadget}
\end{subfigure}
\begin{subfigure}[b]{0.3\textwidth}
\centering
\xymatrix{ 
	\ar @/_/[drr] & \ar @{-} [rrr] \ar @{-} [dd]& & &\ar @{-} [dd] & \ar @/^/[dll]\\
 & & \ar @/_/[dll]&\ar @/^/[drr] & & \\
 & \ar @{-} [rrr]& & & &\\
}	
	\caption{Equality Gadget Representation}
\end{subfigure}
\caption{Equality Gadget Blackbox representation}
\label{fig:EqBlackbox}
\end{figure}

The way we connect a variable and a clause is shown in the next figure. Here $\overline{x}$ is the literal $l_1$ in the clause. 

\begin{figure}[h!]
\centering

\begin{subfigure}[b]{0.3\textwidth}
\centering
\xymatrix{ 
{a}  \ar @/_/ [dd]_x  \ar @/_/[drr]^{u} & \ar @{-} [rrr] \ar @{-} [dd]& & &\ar @{-} [dd] &  & 
{\bullet} \ar @/^/[dlll]^{u'}  \ar @/^/ [d] &  \\ 
 & & \ar @/_/[dll]^{v} &\ar @/^/[drr]^{v'} & &  &  
{\bullet}  \ar @/^/ [u]  \ar @/^/ [dl]  \ar @/^/ [dr] & \\
{b} \ar [uu] & \ar @{-} [rrr]& & & &
{\bullet} \ar @/_/ [rr]^{\overline{l_3}} \ar @/^/ [ur] & & {\bullet} \ar @/_/ [uul]^{\overline{l_2}}  \ar @/^/ [lu] \\	
}	
\end{subfigure}
\caption{Connection between variables and clauses.}
\label{VarClauseConnection}
\end{figure}

When the same variable appears in multiple clauses, we split the edge representing the variable and attach multiple equality gadgets. The next figure shows the split.

\begin{figure}[ht!]
\begin{subfigure}[b]{0.3\textwidth}
\xymatrix{ 
	{\bullet}  \ar @/_/ [ddd]_{\overline{x}}  \ar @/_/ [dr] & & & &  \\ 
& Equality Gadget \ar @/_/ [d] \ar @/_/ [rr] & & Clause Gadget \ar @/_/ [ll]& &\\
& Equality Gadget  \ar @/_/ [dl] \ar @/_/ [rr] & & Clause Gadget \ar @/_/ [ll]& & \\
	{\bullet} \ar [uuu]}
	\label{fig:subfigure1}
\end{subfigure}
	\caption{One variable occurring in multiple clauses.}
\end{figure}

This essentially completes the construction. We will prove the correctness of the construction in a series of claims.
\begin{claim}
Any cycle cover can either use the edge $x$ or $\overline{x}$ in the variable gadget, but not both.
\end{claim}
\begin{proof}
If a cycle cover used both the edges, then the vertices would be covered twice. 
\end{proof}

\begin{claim}
Each assignment corresponds to atleast one cycle cover. 
\end{claim}
\begin{proof}
For each variable, choose $x$ or $\overline{x}$ based on the assignment. And for each clause choose the cycle $l_1 \rightarrow l_2 \rightarrow l_3$ and choose self-loops everywhere else.
\end{proof}

We will derive a much precise correspondence in the remaining proof.


\begin{claim}
In any cycle cover C, either both $u,v$ are used, or neither $u,v$ are used.
\end{claim}
\begin{proof}
The proof is just a verification, We see in \ref{VarClauseConnection} that if the edge $u$ is used, edge $(b,a)$ will have to be used, and to complete a cycle, we will need $v$ to complete the cycle as the edge $x$ cannot be used. 

The proof holds unmodified for edges $u',v'$ too.
\end{proof}

\begin{claim}
If both $u,v$ and $u',v'$ edges are used in the cycle cover, then the $-1$ valued self-loop has to be chosen in the cycle cover. 
\end{claim}

\begin{claim}
If edges $u,v$ are used while $u',v'$ are not used, the corresponding Cycle Covers contribute weight 0.
\end{claim}
\begin{proof}
We just observe that there are two components one contributing $+1$ and the other as $-1$ in the cycle weight. Cycle covers are marked in double lines.

\begin{figure}[H]
\begin{subfigure}[b]{0.3\textwidth}
\centering
 \xymatrix{ 
	\ar @{=>} @/_/[ddr]^{u}  & & {\bullet} \ar@(ul,ur)^{-1} \ar @{=>}@/_/[ddr] \ar   @/_/ [ddl]& & \ar @/^/[ddl]_{u'} \\ 
	\\
	\ar @{<=} @/^/[r]_{v} & {\bullet}  \ar@(dl,dr)_{1}\ar  @/_/ [uur]  \ar @/_/ [rr] & &  {\bullet}  \ar @{=>} @/_/[uul]  \ar @/_/[ll] \ar@(dl,dr)_{1} & \ar @/_/[l]^{v'} }
	\caption{Weight +1}
\end{subfigure}
\begin{subfigure}[b]{0.3\textwidth}
\centering
 \xymatrix{ 
	\ar @{=>}@/_/[ddr]^{u}  & & {\bullet} \ar@{=>}@(ul,ur)^{-1} \ar @/_/[ddr] \ar @/_/ [ddl]& & \ar @/^/[ddl]_{u'} \\ 
	\\
	\ar @{<=}@/^/[r]_{v} & {\bullet}  \ar@(dl,dr)_{1}\ar @/_/ [uur]  \ar @/_/ [rr] & &  {\bullet}  \ar @/_/[uul]  \ar @/_/[ll] \ar@{=>}@(dl,dr)_{1} & \ar @/_/[l]^{v'} }
	\caption{Weight -1}
\end{subfigure}
\caption{Only one of the two sets of edges are present}
\end{figure}
\end{proof}


\begin{claim}
If both $u,v$ and $u',v'$ are not used, then the Cycle Covers have a contribution of 2 from this gadget. 
\end{claim} 
\begin{proof}
The figure \ref{Fig6} contains all the possible cycle covers of the gadget. Their contributions sum upto 2.
\begin{figure}[h]
\begin{subfigure}[b]{0.3\textwidth}
\centering
 \xymatrix{ 
	\ar  @/_/[ddr]^{u}  & & {\bullet} \ar @{=>}@(ul,ur)^{-1} \ar @/_/[ddr] \ar   @/_/ [ddl]& & \ar @/^/[ddl]_{u'} \\ 
	\\
	\ar  @/^/[r]_{v} & {\bullet}  \ar @{=>}@(dl,dr)_{1}\ar  @/_/ [uur]  \ar @/_/ [rr] & &  {\bullet}  \ar @/_/[uul]  \ar @/_/[ll] \ar @{=>}@(dl,dr)_{1} & \ar @/_/[l]^{v'} }
	\caption{Weight -1}
\end{subfigure}
\begin{subfigure}[b]{0.3\textwidth}
\centering
 \xymatrix{ 
	\ar @/_/[ddr]^{u}  & & {\bullet} \ar@{=>}@(ul,ur)^{-1} \ar @/_/[ddr] \ar @/_/ [ddl]& & \ar @/^/[ddl]_{u'} \\ 
	\\
	\ar @/^/[r]_{v} & {\bullet}  \ar@(dl,dr)_{1}\ar @/_/ [uur]  \ar@{=>}@/_/[rr] & &  {\bullet}  \ar @/_/[uul]  \ar@{=>}@/_/[ll] \ar@(dl,dr)_{1} & \ar @/_/[l]^{v'} }
	\caption{Weight -1}
	\end{subfigure}
\begin{subfigure}[b]{0.3\textwidth}
\centering
 \xymatrix{ 
	\ar  @/_/[ddr]^{u}  & & {\bullet} \ar @(ul,ur)^{-1} \ar @{=>}@/_/[ddr] \ar   @/_/ [ddl]& & \ar @/^/[ddl]_{u'} \\ 
	\\
	\ar  @/^/[r]_{v} & {\bullet}  \ar @{=>}@(dl,dr)_{1}\ar  @/_/ [uur]  \ar @/_/ [rr] & &  {\bullet}  \ar @{=>}@/_/[uul]  \ar @/_/[ll] \ar @(dl,dr)_{1} & \ar @/_/[l]^{v'} }
	\caption{Weight 1}
\end{subfigure}	
\begin{subfigure}[b]{0.3\textwidth}
\centering
 \xymatrix{ 
	\ar  @/_/[ddr]^{u}  & & {\bullet} \ar @(ul,ur)^{-1} \ar @/_/[ddr] \ar@{=>}@/_/ [ddl]& & \ar @/^/[ddl]_{u'} \\ 
	\\
	\ar  @/^/[r]_{v} & {\bullet}  \ar@(dl,dr)_{1}\ar  @{=>}@/_/ [uur]  \ar @/_/ [rr] & &  {\bullet}  \ar @/_/[uul]  \ar @/_/[ll] \ar @{=>}@(dl,dr)_{1} & \ar @/_/[l]^{v'} }
	\caption{Weight 1}
\end{subfigure}	
\begin{subfigure}[b]{0.3\textwidth}
\centering
 \xymatrix{ 
	\ar  @/_/[ddr]^{u}  & & {\bullet} \ar @(ul,ur)^{-1} \ar @/_/[ddr] \ar@{=>}@/_/ [ddl]& & \ar @/^/[ddl]_{u'} \\ 
	\\
	\ar  @/^/[r]_{v} & {\bullet}\ar@(dl,dr)_{1}\ar@/_/ [uur]\ar@{=>}@/_/ [rr] & &  {\bullet}  \ar@{=>}@/_/[uul]  \ar @/_/[ll] \ar @(dl,dr)_{1} & \ar @/_/[l]^{v'} }
	\caption{Weight 1}
\end{subfigure}	
\begin{subfigure}[b]{0.3\textwidth}
\centering
 \xymatrix{ 
	\ar@/_/[ddr]^{u}  & & {\bullet} \ar@(ul,ur)^{-1} \ar@{=>}@/_/[ddr] \ar@/_/ [ddl]& & \ar @/^/[ddl]_{u'} \\ 
	\\
	\ar  @/^/[r]_{v} & {\bullet}\ar@(dl,dr)_{1} \ar@{=>}@/_/[uur] \ar@/_/[rr] & &  {\bullet}\ar@/_/[uul]  \ar@{=>}@/_/[ll] \ar@(dl,dr)_{1} & \ar@/_/[l]^{v'} }
	\caption{Weight 1}
\end{subfigure}	
\caption{The weights sum upto 2}
\label{Fig6}
\end{figure}
\end{proof}

\begin{claim}
	Weight of each cycle cover is $(-2)^{kn}$
\end{claim}
\begin{proof}
We want to claim that if a variable $x$ is assigned value $1$, then all the equality gadgets for $x$ will contribute $1$ to the weight because $u,v$ and $u',v'$ will both be a part of the cycle cover for each of the gadgets, if just one pair is in the cycle cover, we have seen that those covers would contribute 0 to the weight. 

Also, the gadgets that correspond to $\overline{x}$ will not have either $u,v$ or $u',v'$ being used - because, $u,v$ cannot be used as $x$ edge will be used, hence $\overline{x}$ cannot be used. Now, if $u',v'$ are used, those cycles will have 0 weight as seen in the observation.

So, the only possibility there is both $u,v$ and $u',v'$ are not used. In that case, the contribution would be $2$ for each gadget. As there are $k$ such gadgets, we will have a contribution of $2^k$ from these. 

So, multiplying them would give us, that each variable pair $x,\overline{x}$ contribute exactly $(-2)^k$ to the weight. Hence the cycle cover would have a weight of exactly $(-2)^{kn}$.

So, We sum them up over all the possible assignments to get the required result. This completes the proof.

\end{proof}

So, we have the result -
\begin{equation}
{\sum_{\text{C is a Cycle Cover}} \wt(C) }= \per_{-1,0,1}(A)
\end{equation}

Hence computing $\#\SAT$ reduces to computing $\per_{-1,0,1}$. This completes the proof. 
\end{proof}

Now we will first show that $\per_{-1,0,1}$ reduces to $\per_{0,1,\dots n}$ and finally show that $\per_{0,1,\dots n}$ reduces to $\per_{0,1}$ and hence completing the theorem.

\begin{theorem}
$\per_{-1,0,1} \in \FP^{\per_{0,1,\dots n}}$.
\end{theorem}
\begin{proof}
 The first thing we observe is that all the -1 terms in the adjacency matrix $A$ represent self-loops because in the construction, $-1$ was the edge weight of only one self-loop.
 
 Consider $\per(A)$ as a polynomial in $x$ where each $-1$ is replaced by $x$ denoted by $p(x)$ 
 
 Now we just observe that, using ${\per_{0,1,\dots n}}$ as an oracle, we can find $p(0), p(1), ... p(n)$. Also, degree of p $\leq n$ because x occurs only in the diagonal entries, hence only n $x $ can be present. 
 
 So, now we just use Lagrange Interpolation to find the polynomial $p$. This can be done in poly time. Once this is done, $\per_{-1,0,1}(A) = p(-1)$, which can be computed easily.
 
 This completes the reduction.
\end{proof}

In the final reduction, we show that $\per_{0,1,\dots n} \in \FP^{\per_{0,1}}$, and that $\per_{0,1}$ is as hard as the other $\per$ computations. 

\begin{theorem}
$\per_{0,1,\dots n} \in \FP^{\per_{0,1}}$
\end{theorem}
\begin{proof}
This proof involves substituting the $-1$ self-loop with a gadget so that we can compute the values of the polynomial $p(x)$ at points $x=0, x=1, \dots x=n$.

The gadget we use is - 
Consider any $a = (a_k, a_{k-1}\dots a_{0})_2$ in base $2$ where $a\in \{0, 1, \dots n\}$. Now we want to replace the self-loop of weight $a$ with the gadget, so that the gadget contributes exactly $k$ weight to the Cycle cover. 

\begin{figure}[ht!]
\xymatrix{
\ar[r]^{1} & \ar@/^/[d]^{a_0} \ar@/_/[r]^{1} \ar@/^/[r]^{1} & \ar@(ul,ur) \ar@/^/[dl]^{a_1} \ar@/_/[r]^{1} \ar@/^/[r]^{1}  & \ar@(ul,ur) \ar@/^/[dll]^{a_2} \ar@[--][r] & \ar@(ul,ur) \ar@/^/[dlll]^{a_{k-1}}   \ar@/_/[r]^{1} \ar@/^/[r]^{1} & \ar@(ul,ur) \ar@/^/[dllll]^{a_{k}} \\ 
& \ar[ul]^{1} & & \\
}
\end{figure}

The gadget has precisely $n$ cycles of weight $1$ each. This gadget can be used to replace the self loop and then query the $\per_{0,1}$ oracle. 

This completes the reduction. Hence proved.
\end{proof}

